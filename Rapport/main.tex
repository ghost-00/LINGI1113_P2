\documentclass{article}

\usepackage[utf8x]{inputenc}
\usepackage[frenchb]{babel}
\usepackage[T1]{fontenc}
\usepackage{lmodern}
\usepackage{fullpage}
\usepackage{graphicx}
\usepackage{epstopdf}
\usepackage{caption}
\usepackage{subcaption}
\usepackage{multirow}
\usepackage{graphicx} %intégration d'image

% Math symbols
\usepackage{amsmath}
\usepackage{amssymb}
\usepackage{amsthm}

% Numbers and units
\usepackage[squaren, Gray]{SIunits}
\usepackage{sistyle}
\usepackage[autolanguage]{numprint}
%\usepackage{numprint}
\newcommand\si[2]{\numprint[#2]{#1}}
\newcommand\np[1]{\numprint{#1}}

\DeclareMathOperator{\newdiff}{d} % use \dif instead
\newcommand{\dif}{\newdiff\!}
\newcommand{\fpart}[2]{\frac{\partial #1}{\partial #2}}
\newcommand{\ffpart}[2]{\frac{\partial^2 #1}{\partial #2^2}}
\newcommand{\fdpart}[3]{\frac{\partial^2 #1}{\partial #2\partial #3}}
\newcommand{\fdif}[2]{\frac{\dif #1}{\dif #2}}
\newcommand{\ffdif}[2]{\frac{\dif^2 #1}{\dif #2^2}}
\newcommand{\constant}{\ensuremath{\mathrm{cst}}}

\newcommand{\bigoh}{\mathcal{O}}

% cfr http://en.wikibooks.org/wiki/LaTeX/Colors
\usepackage{color}
\usepackage[usenames,dvipsnames,svgnames,table]{xcolor}
\definecolor{dkgreen}{rgb}{0.25,0.7,0.35}
\definecolor{dkred}{rgb}{0.7,0,0}

\usepackage{listings}
\lstset{
  numbers=left,
  numberstyle=\tiny\color{gray},
  basicstyle=\rm\small\ttfamily,
  keywordstyle=\bfseries\color{dkred},
  frame=single,
  commentstyle=\color{gray}=small,
  stringstyle=\color{dkgreen},
  %backgroundcolor=\color{gray!10},
  %tabsize=2,
  rulecolor=\color{black!30},
  %title=\lstname,
  breaklines=true,
  framextopmargin=2pt,
  framexbottommargin=2pt,
  extendedchars=true,
  %inputencoding=utf8x
}

\usepackage{hyperref}

\lstset{language={Java}}

\title{Système informatique 2 \\ Projet 2 : Programmation sur machine nue \\ Groupe 23}
\author{Mehdi Dumoulin 45570900 \and Jos zigabe 63271000}

\begin{document}

\maketitle
\tableofcontents %table de matière cliquable
\newpage % saut de page

\part*{Rapport projet 2}
	\section{Mode d'emploi}

	Le programme commence à 00:00:00 et se lance automatiquement.
A l'aide du \texttt{bouton 1}  l'utilisateur pourra naviguer entre les différents éléments proposés et grâce au \texttt{bouton 2} il pourra effectuer des sélections.\\
	
	\subsubsection*{Mode $Accueil$} 
Au démarrage le mode \texttt{ACCUEIL} est celui qui est proposé à l'utilisateur et dans ce mode l'horloge est affichée de manière classique.\\

Dans un premier temps, un appui sur le \texttt{bouton 2} active le programme et un clignotement indiquera la sélection courante et un appuie sur le \texttt{bouton 1} lui permettra de naviguer entre $Time$ et $Alarm$. A ce niveau du programme, si l'utilisateur appuie de nouveau sur le \texttt{bouton 2}, il entrera dans le mode réglage de l'horloge (\texttt{edit\_T}) ou le mode réglage de l'alarme (\texttt{edit\_A}) en fonction du clignotement actuel. \\
	Ensuite dans ce mode \texttt{ACCUEIL} l'utilisateur pourrait également appuyer sur le \texttt{bouton 1} et entrer dans ce cas dans un mode que l'on a intitulé \texttt{HELLO} qui affiche les infos relative à l'alarme et un appui sur l'un des deux boutons quittera ce mode. 
	
	\subsubsection*{Mode réglage $Time$/$alarm$} 
Qu'on soit dans le mode $Time$ ou $Alarm$, le \texttt{bouton 2} servira à naviguer dans l'heure et le  \textbf{bouton 1} servira à modifier l'élément sélectionné.\\

Dans le mode réglage de l'alarme, il y a la possibilité d'activer ou de désactiver l'alarme en cliquant sur $N/Y$ (No/Yes) et ensuite il faut sélectionner l'heure et la minute à laquelle l'alarme va s'activer.
Dans le mode réglage de l'heure, on a la possibilité d'avancer ou de reculer l'heure.  Après avoir réglé les secondes, lors de l'appui sur le bouton de selection (\texttt{bouton 2}), l'horloge se remet en mode veille et le programme retourne ainsi en mode \texttt{ACCUEIL}.

	\section{installation}
		
		Pour installer le programme sur le PIC, ouvrez un terminal et situez vous dans le dossier où se trouve l'application. Ensuite tapez la commande \texttt{\$make pic}, le programme se compile et un fichier $pic.hex$ est créé.
Pour envoyer le fichier sur le PIC tapez la commande  \texttt{\$tftp 192.168.97.60}.
Ensuite, tapez la commande \texttt{\$binary},  puis \texttt{\$trace}, suivie de \texttt{\$verbose}.
Enfin, appuyer sur le bouton $reset$ du PIC et tapez la commande \texttt{\$put pic.hex} et le programme sera envoyé au PIC.

	\section{documentation}
		\subsection{Spécification}
		Notre programme permet d'afficher l'heure en mode 24h sous le format \texttt{hh:mm:ss}. L'affichage est mis à jour toutes les secondes et l'utilisateur à la possibilité à tout instant de modifier l'heure et de programmer une alarme. La sonnerie de l'alarme étant un clignotement du \texttt{LED01} pendant 30 secondes mais cette alarme peut évidemment être stop avant que les 30 secondes ne soient écoulées.   
		
		\subsection{choix de structure du programme}
		Notre programme fonctionne par interruptions et la méthode \texttt{main} ne sert qu'à configurer les différents paramètres tels que initialiser \texttt{0} et mettre les \texttt{LED} en output ou activer les interruptions et mettre les \texttt{bouton 1} et \texttt{bouton 2} en input. Et dans notre programme on peut identifier trois sortes d'interruptions :\\
		\begin{enumerate}
		\item L'interruption du \texttt{timer} qui s'exécute à chaque seconde
		\item L'interruption due à l'appui du \texttt{bouton 1}
		\item L'interruption due à l'appui du \texttt{bouton 2}\\
		\end{enumerate}
		
		Notre programme peut se retrouver dans 7 \texttt{Mode} differents et les boutons agirons en fonctions du mode dans lequel ils se trouvent  :\\
		
		\begin{tabular}{|p{2cm}|p{13cm}|}
		\hline
		\textbf{Mode} & \textbf{Description}\\
		\hline
		\multirow{3}{*}{\texttt{ACCUEIL}} & Dans ce mode l'heure est affiché de manière classique, c'est l'interface principale.\\
		                         &\texttt{bouton 1} : Dans cet interface, une interruption du bouton 1 permet d'affiche l'interface \texttt{HELLO} séquentiellement.\\
		                         &\texttt{bouton 2} : Une interruption du bouton 2 permettra d'accéder au mode \texttt{Time} qui fera clignoter le $Time$ sur l'écran\\ 
		\hline
		\multirow{3}{*}{\texttt{TIME}} & Dans cette interface, le $Time$ sur l'écran LCD clignote en permanence.\\
		                                              & \texttt{bouton 1} : Une interruption sur le bouton 1 permet de passer au mode \texttt{ALARM}.\\ 
		                                              &\texttt{bouton 2} : Une interruption sur le bouton 2 permet de passer au mode \texttt{EDIT\_T} qui fera d'ailleurs clignoter le premier élément de l'heure.\\ 
		\hline
		\multirow{3}{*}{\texttt{ALARM}} & Dans cette interface, le $Alarm$ sur l'écran LCD clignote en permanence.\\
													    & \texttt{bouton 1} : Une interruption sur le bouton 1 permet de passer au mode \texttt{TIME}.\\ 
													    & \texttt{bouton 2} : Une interruption sur le bouton 2 permet de passer au mode \texttt{EDIT\_A} qui fera clignoter l'état actuel du réveil \texttt{Y or N}.\\ 
		\hline
		\multirow{3}{*}{\texttt{EDIT\_T}} & Cette interface fait clignoter l'élément courant.\\
														& \texttt{bouton 1} : un clique sur le bouton 1 permet d'incrémenter de 1 l'élément actuel.\\ 
														& \texttt{bouton 2} : un clique sur le bouton 2 fait appel à une fonction \texttt{move()} qui incrémente de 1 \texttt{posC} afin de pouvoir passer à l'élément suivant.\\    
		\hline
		\multirow{3}{*}{\texttt{EDIT\_A}} & Ce mode fait clignoter l'état actuel du $réveil$\\
														& \texttt{bouton 1} : un clique sur le bouton 1 nous permettra d'activer ou de désactiver le réveil.\\
														& \texttt{bouton 2} : un clique sur le bouton 2 fait appel à une fonction \texttt{move()} qui incrémente de 1 \texttt{posA} afin de pouvoir passer à l'élément suivant.\\ 
		\hline
		\multirow{2}{*}{\texttt{HELLO}} & Ce mode affiche séquentiellement les infos relatives au réveil.  \\
													   & Un clique sur l'un des deux boutons nous faire retourner au mode principale.\\
		\hline
		\multirow{2}{*}{\texttt{RINGTONE}} & Ce mode ne fait que afficher un message lorsque l'alarme sonne et ne s'active que à cet instant. \\
															 & Un clique sur l'un des deux boutons nous faire retourner au mode principale.\\
		\hline
		\end{tabular}
		
		 
		\subsection{Détails techniques du PIC}

\end{document}
